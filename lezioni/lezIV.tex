\chapter[Lezione IV]{Lezione IV\newline\small{\emph{14/04/2011}}}
	\section{Esercizio}
Si vuole calcolare il valore di $\pi$. \`E noto che:
\[
\frac{\pi}{4}=\sum_{k=0}^{+\infty}(-1)^k\frac{1}{2k+1}=1-\frac{1}{3}+\frac{1}{5}\dots
\]
Questo metodo non può essere steso come \emph{algoritmo} (che è \textbf{finito} per definizione\footnote{Un algoritmo è un procedimento che consente di ottenere un risultato eseguendo, in un determinato ordine, dei passi semplici (scelti da un insieme finito). Pertanto:
\begin{itemize}[noitemsep]
	\item
La sequenza di istruzioni deve essere finita (\emph{finitezza});
	\item
Essa deve portare ad un risultato (\emph{effettività});
	\item
Le istruzioni devono essere eseguibili materialmente (\emph{realizzabilità});
	\item
Le istruzioni devono essere espresse in modo non ambiguo (\emph{non ambiguità}).
\end{itemize}   
I passi costituenti di un algoritmo devono essere ""semplici'', nel senso di ""non ambigui''.}). Tuttavia, ci si può accontentare di approssimazioni di $\pi$ che discostino dal valore reale per una quantità finita piccola a piacere facendo eseguire al calcolatore un numero finito (se pur molto grande) di iterazioni. La sommatoria, allora, diverrà:
\[
\frac{\pi}{4} \thickapprox \sum_{k=0}^{n\in\mathbb{N}}(-1)^k\frac{1}{2k+1}=1-\frac{1}{3}+\frac{1}{5}\dots+\frac{\ (-1)^n}{2n+1}.
\]

Un algoritmo come quello del codice~\vref{cod:pigreco} calcolerà un valore approssimato del valore di $\pi$. Chiaramente, la precisione del valore calcolato dipenderà dalla grandezza di $n$. Infatti
\[
\lim_{n\to\infty}\sum_{k=0}^n\frac{\ (-1)^k}{2k+1} =\frac{\pi}{4}.
\]
\begin{lstlisting}[caption={\em Calcolo del valore approssimato di $\pi$.}, label={cod:pigreco}]
#include <stdio.h>
#include <stdlib.h>
#include <math.h>

int main ( int argc, char *argv[] ) {
	int n, i = 0;
	double somma = 0.0;
	printf("Inserisci n: ");
	scanf("%d", &n);
	while ( i <= n ) {
		if ( i%2 == 0 )
			somma = somma + (1/(2*i + 1));
		else
			somma = somma - (1/(2*i + 1));
		i = i + 1;
	}
	printf("Pi Greco = %lf", somma*4);
	exit(EXIT_SUCCESS);
}
\end{lstlisting}
