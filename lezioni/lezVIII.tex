\chapter[Lezione VIII]{Lezione VIII\newline\small\emph{12/05/2010}}
	\section{Costruzione di una lista concatenata}
Come già accennato, una lista concatenata è una sequenza di oggetti (chiamati anche \emph{nodi}) nella quale è definita una relazione d'ordine. Il criterio d'ordinamento è dato dalla successione dei puntatori. In una lista concatenata, per raggiungere l'n-esimo elemento bisogna percorrere tutti gli $n-1$ elementi precedenti.

Per costruire concretamente una lista concatenata c'è bisogno, oltre che degli elementi della lista, di una variabile (puntatore) che contenga l'indirizzo del primo elemento. In generale, risulta comodo chiamare questa variabile \lstinline!*testa!, come nel codice~\vref{cod:ListaConc}. All'inizio la lista è vuota ma \lstinline!testa! deve avere un valore legittimo. Per convenzione si pone, in questo caso, \lstinline!testa = NULL!. Si noti che l'assegnamento \lstinline!(*pn).next = testa;! equivale a \lstinline!pn->next = testa;!: usare una delle due sintassi è indifferente ai fini della compilazione del programma.
\begin{lstlisting}[caption={\em Costruzione di una lista concatenata.}, label={cod:ListaConc}]
struct nodo {
	int v;
	char c;
	struct nodo *next;
} *testa, *pn;

typedef struct nodo Nodo;
testa = NULL;

pn = (Nodo *)malloc(sizeof(Nodo));
testa = pn;
int i = 1;
while ( i < 10 ) {
	pn = (Nodo *)malloc(sizeof(Nodo));	
	if ( pn == NULL )
		exit(EXIT_FAILURE);
	(*pn).next = testa;
	testa = pn;
	i++;
} 
\end{lstlisting}

A differenza dei vettori, i nodi della lista concatenata non sono indicizzati tramite dei numeri ma, come già accennato, tramite dei puntatori. Volendo, allora, scorrere tutti gli oggetti presenti fino all'ultimo, si scriverà un codice simile al seguente:
\begin{lstlisting}
pn = testa;
while ( pn != NULL ) {
	/* eventuali istruzioni */
	pn = pn->next;
}
\end{lstlisting}
Si tenga presente che, se la lista è vuota, il ciclo fallisce perché la condizione \lstinline!pn == NULL! è verificata fin dal primo elemento.

	\section{Esempi}
I codici~\vref{tab:begin}, ~\vref{tab:end} e~\vref{tab:middle} sono esempi di come inserire nodi in una lista concatenata rispettivamente all'inizio, alla fine ed in una posizione intermedia.\footnote{Nell'esempio~\ref{tab:middle} non è stata inserita la \lstinline!/* condizione */! perché dipende dal criterio che si vuole usare per inserire il nodo.}
	
\begin{lstlisting}[caption={\emph{Inserimento in testa}}, label={tab:begin}]
pn = (Nodo *)malloc(sizeof(Nodo));
if ( ... )
	exit(...);
pn->next = testa;
testa = pn;
\end{lstlisting}

\begin{lstlisting}[caption={\emph{Inserimento in coda}}, label={tab:end}]
/* pi funge da contatore */
pi = testa;
/* scorro la lista finche' non finisce */
while( pi->next != NULL)
	pi = pi->next;
/*
** assegno all'ultimo puntatore l'indizizzo
** del nodo pn (creato con malloc();)
*/
pi->next = pn;
/*
** Ora pn e' l'ultimo, quindi il suo campo
** .next dev'essere NULL
*/
pn->next = NULL;
\end{lstlisting}

\begin{lstlisting}[caption={\emph{Inserimento intermedio}}, label={tab:middle}]
pi = testa;
while( pi->next /* condizione */)
	pi = pi->next;

pn->next=pi->next;
pi->next = pn;
\end{lstlisting}
	

