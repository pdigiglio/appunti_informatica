\chapter[Lezione V]{Lezione V\newline\small{\emph{21/04/2011}}}
	\section{Ancora sulle funzioni}
	\label{sec:fun2}
La funzione è una specie di ``scatola'' cui si associa un nome simbolico: vi entrano dei dati e ne esce il risultato della loro elaborazione.
Quest'ultima avviene per mezzo delle istruzioni che ne compongono il corpo.

In \marginpar{L'istruzione \lstinline!return!} una funzione, l'istruzione \lstinline!return! è sempre l'ultima ad essere eseguita e ne indica la fine.
Si noti che ciò non vuol dire essa che debba essere l'ultima ad essere scritta.
Se è inclusa in una scelta, ad esempio, non è affatto detto che debba essere eseguita.
\`E lecito, dunque, che dopo ``\lstinline$return $\MyComment{\dots}\lstinline$;$'' vi siano altre righe di codice.

Quando una funzione viene eseguita, il processo principale si ``blocca'' in attesa del suo risultato.
Tutte le variabili create durante questo lasso di tempo vengono cancellate quando essa termina (all'istruzione \lstinline!return!, appunto). 

Nella\marginpar{Argomenti formali e reali} definizione della funzione, le variabili che si trovano all'interno delle parentesi tonde sono dette \emph{argomenti formali}\index{argomento formale}.
Nel momento in cui la si richiama in qualche altra riga del codice, le si passano degli \emph{argomenti reali}\index{argomento reale}.
Il codice~\ref{cod:callfunc} è un esempio di chiamata della funzione \lstinline!flip()! (all'interno della funzione \lstinline!main()!).
Qui, le variabili \lstinline!a!, \lstinline!b! e \lstinline!c! sono gli argomenti reali della chiamata della funzione \lstinline!flip()!.
Si noti che essi sono stati dichiarati in modo da essere compatibili con i tipi richiesti dagli argomenti formali.
\begin{lstlisting}[caption={\em Chiamata della funzione \lstinline!flip()!.}, label={cod:callfunc}]
double flip (int x, double y, int z ) {

	£!\MyComment{istruzioni}!£

	return £!\MyComment{\dots}!£;
}

int main ( int argc, char *argv[] ) {
	int a, c;
	double b, x;

	x = flip( a, b, c );

	£!\MyComment{\dots}!£
}
\end{lstlisting}


Le\marginpar{Variabili locali e globali} variabili definite in qualsiasi funzione prendono il nome di \emph{variabili locali}\index{variabile!locale} della funzione \lstinline!x()! (dove \lstinline!x()! è il nome della stessa: \lstinline!flip()!, ad esempio).
In \lang{C}, esse si distinguono dalle  \emph{variabili globali}\index{variabile!globale} che, dichiarate al di fuori di tutte le funzioni, restano a disposizione di ognuna di esse per tutta la durata del programma---cioè della funzione \lstinline!main()!.
Ogni funzione può modificare il valore di una variabile globale ma non è possibile richiamare direttamente una variabile della funzione \lstinline!flip()! dalla funzione \lstinline!main()!, ad esempio.\footnote{Lo si può però fare tramite i puntatori (vedi il paragrafo~\ref{sec:pointers} a pagina~\pageref{sec:pointers}).}

La\marginpar{Passaggio per valore} comunicazione tra funzioni avviene tramite un procedimento chiamato \emph{passaggio per valore} (descritto nel prossimo esempio).
Una funzione riceve il valore di una variabile come parametro reale, ma \emph{non} può modificare il valore assegnato alla stessa.\footnote{Questo non è sempre vero. A causa della loro stretta relazione coi puntatori, è possibile che il valore di una variabile dichiarata come array passata come argomento reale venga modificata durante l'esecuzione di una funzione (si veda sempre il paragrafo~\ref{sec:pointers} a pagina~\pageref{sec:pointers}).}
Quando una funzione viene eseguita, il calcolatore le riserva uno spazio chiamato \emph{record di attivazione}\index{record di attivazione}.

Si supponga di voler scrivere un programma che calcola la somma dei primi $n$ elementi di un array:
\begin{lstlisting}
double somma ( double v[], int n ) {
	int i = 0;
	double s = 0.0;

	while ( i < n) {
		s = s + v[i];
		i = i + 1;
	}

	return s;
}

int main ( int argc, char *argv[] ) {
	double m[10], p[20], v, w;

	v = somma( m, 10 );
	w = somma( p, 15 );

	£!\MyComment{\dots}!£

}
\end{lstlisting}
Si noti che tale programma funziona se e solo se, quando si chiama la funzione \lstinline!somma()!, il secondo argomento è minore o uguale alla lunghezza del vettore passato come primo argomento. In caso contrario, si otterrà soltanto un \emph{errore di segmentazione}.


\begin{code}
\begin{minipage}{0.45\columnwidth}
	\begin{lstlisting}[caption={\em Il tipo \lstinline!char!.},nolol,label={code:char}]
char x;
char p[10];
p[3] = 'n';
	\end{lstlisting}
\end{minipage}	\hfill
\begin{minipage}{0.45\columnwidth}
	\begin{lstlisting}[caption={\ },nolol,label={code:Printb}]
char x = 'a';
printf("%c", x + 1);
	\end{lstlisting}
\end{minipage}
\caption{I caratteri in \lang{C}.}
\label{riq:char}
\end{code}
Il codice~\ref{code:char} nel riquadro~\ref{riq:char}, introduce ora un nuovo tipo di variabile, che corrisponde al “simbolo”.
I vettori di caratteri costituiscono la rappresentazione delle parole in \lang{C}. 
Le variabili di tipo \lstinline!char! si dichiarano racchiudendo il valore da assegnare tra singoli apici. Anche i numeri possono essere considerati dei caratteri, purché racchiusi tra apici.
Dopo aver dichiarato un numero come carattere, tuttavia, non è possibile eseguire le consuete operazioni algebriche su di esso.

Nel \marginpar{Rappresentazione dei caratteri} calcolatore, ad ogni carattere è assegnata una sequenza di cifre binarie (in genere, \num{8} cifre) che è effettivamente un numero.
Pertanto, l'espressione ``\lstinline!p[3] * 4!'' ha significato, ma non è quello che ci si aspetterebbe.
La codifica numerica dei caratteri, tuttavia, può consentire delle operazioni interessanti.
Le lettere dell'alfabeto, ad esempio, sono numerate in ordine progressivo.\footnote{Le lettere maiuscole sono poste in ordine progressivo tra di loro, così come quelle minuscole. Tuttavia, le rappresentazioni di \lstinline!a! e \lstinline!B! non hanno differiscono tra di loro per un unità.}
Con un programma simile a quello del codice~\ref{code:Printb}, il calcolatore stamperà a schermo il carattere ``b''.
Tale rappresentazione dei caratteri permette, inoltre, di ordinare alfabeticamente le parole.

		\subsection{Filtri}
L'istruzione\marginpar{La funzione \lstinline!getchar()!} ``\lstinline!x = getchar();!''\index{getchar()@\texttt{getchar()}} comunica all'esecutore di leggere il carattere successivo dallo standard input (in genere, la tastiera), assegnarlo alla variabile \lstinline!x! e di toglierlo dalle serie dei ``caratteri in entrata''.
La funzione \lstinline!getchar()! non ammette parametri e ha come unico risultato l'assegnamento del carattere letto alla variabile cui è associata (nella fattispecie, \lstinline!x!).
L'\marginpar{La funzione \lstinline!putchar()!}istruzione ``\lstinline!putchar(x);!''\index{putchar()@\texttt{putchar()}}, invece, stampa sullo \emph{standard output}\index{standard output} (\lstinline!stdout!\index{stdout@\texttt{stdout}}, in genere il monitor) il valore della variabile \lstinline!x!.

Il programma nel codice~\vref{cod:eco} ``fa l'eco'' di quanto riceve dallo standard input.
Esso appartiene alla famiglia dei \emph{filtri}, cioè programmi che leggono dei caratteri ed eseguono delle operazioni su di essi.
Il valore \lstinline!EOF! è una sequenza di caratteri speciale (del linguaggio \lang{C}) che sta per \emph{End Of File}.
Per i sistemi \os{UNIX}, la sequenza riconosciuta come \lstinline!EOF! è la combinazione di tasti \key{CTRL + D}. 
\begin{lstlisting}[caption={\em Esempio di filtro.}, label={cod:eco}]
#include <stdio.h>
#include <stdlib.h>
#include <ctype.h>

int main ( int argc, char *argv[] ) {
	int c;
	/* leggo il primo carattere */
	c = getchar();
	while ( c != EOF ) { /* fino a quando non premo 'CTRL + D' */
		/* stampo il carattere */
		putchar(c);
		/* leggo il successivo */
		c = getchar();
	}

	exit(EXIT_SUCCESS);
}
\end{lstlisting}