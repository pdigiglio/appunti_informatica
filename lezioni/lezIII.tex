\chapter[Lezione III]{Lezione III\newline\small{\emph{07/04/2011}}}
	\subsection{Iterazione}
	\label{sec:it}


Nel linguaggio \lang{C} una maniera per introdurre un'\emph{iterazione}\index{iterazione} è la locuzione
\begin{lstlisting}
while ( £!\MyComment{test}!£ ) {
	£!\MyComment{\dots}!£
}
\end{lstlisting}
Si noti che <<iterare>> vuol dire <<ripetere>> quindi il corpo dell'iterazione verrà ri-eseguito finché il \MyComment{test} non diventa falso.
In questo caso l'esecutore salta il blocco \lstinline!while! e procede con il comando che segue.
Esistono due tipi di iterazioni o, più esattamente, condizioni verificate le quali un'iterazione continua:
\begin{itemize}
	\item
L'iterazione viene ripetuta un numero $n$ di volte fissato a priori;
	\item
Il numero di volte $n$ per cui si ripete l'operazione viene determinato da come procede l'esecuzione (o l'iterazione stessa).
\end{itemize}


\begin{lstlisting}[
	caption={{Tabella di conversione Euro/Lira.}},
	label={cod:conv},
	float
]
/* dichiaro la variabile (contatore) e la inizializzo */
int i = 0; 
while ( i < 10 ) {
	printf("%d ... %f", i, i * 1936.27);

	/* incremento il valore del contatore */
	i = i + 1; 
}
\end{lstlisting}
Nel linguaggio \lang{C}, non esiste una forma che equivalga alla locuzione <<ripeti l'operazione per $n$ volte>> ma bisogna ricorrere ad un'iterazione regolata da una determinata condizione.
Un espediente piuttosto frequente è mostrato nel codice~\ref{cod:conv}: si dichiara una variabile di tipo \lstinline!int! e la si incrementa ad ogni iterazione in modo che funga da \emph{contatore}\index{contatore}\marginpar{Contatore}.
Pertanto per eseguire \num{10} volte le istruzioni contenute nel corpo si controlla di volta in volta che il contatore sia minore di \num{10}.
Quando il contatore \lstinline!i! nel codice~\ref{cod:conv} varrà \num{10} il test non sarà più vero e l'iterazione, o \emph{ciclo}\index{ciclo}, verrà interrotta.


Un errore frequente è impostare un ciclo \lstinline!while! dimenticandosi d'incrementare il contatore.
Il risultato è un ciclo infinito senza \emph{alcuna} possibilità per il calcolatore di segnalare errori di questo tipo (si veda il paragrafo~\ref{subsec:ins}).


Nel codice~\ref{cod:conv} è stato\marginpar{Commenti} introdotto l'uso dei \emph{commenti}\index{commento}, parti di codice che il compilatore ignora.
I commenti servono per migliorare la leggibilità del sorgente e sono compresi tra \lstinline!/* slash ed asterischi */! oppure introdotti da un \lstinline!// doppio slash!.
Nella prima forma il commento finisce dopo il secondo asterisco-slash mentre nella seconda tutti i caratteri sulla stessa riga, da \lstinline!//! in poi, vengono ignorati al momento dell'esecuzione.


Il frammento di codice~\ref{cod:conv} è una tabella di conversione tra Euro e Lire.
Se opportunamente completato, stampa a schermo due colonne: a sinistra un numero intero da \numrange{0}{9} e a destra lo stesso numero moltiplicato per \num{1936.27}.
\begin{table}
	\caption[Sintassi della funzione \lstinline!printf()!]{Possibili sintassi della funzione \lstinline[mathescape]!printf($\text{\MyComment{entry}}$)!.
Ogni sotto-espressione introdotta dal carattere \lstinline!\%!
indica il punto in cui sarà inserito il valore di una variabile.
La sequenza tra virgolette è seguita dall'elenco delle variabili cui fa riferimento separate da virgole.
}
	\label{tab:printf}
	\centering
	\begin{tabular}{Fp{0.67\columnwidth}}
		\toprule
$\text{\MyComment{entry}}$	& Funzione \\
		\midrule
"testo"
& Stampa il messaggio racchiuso tra virgolette letteralmente.
Usando la combinazione \lstinline$"testo\n"$ si andrà a capo dopo la parola <<testo>>. \\

"\%d", num
& Stampa il valore della variabile \lstinline$num$, che dev'essere di tipo \lstinline$int$. \\

"\%f", x
&  Stampa il valore della variabile \lstinline$x$ di tipo \lstinline!float!; per un \lstinline!double! l'identificatore è \lstinline!%lf!. \\

"text \%8.2f", x
&La combinazione \lstinline$%$%
\MyComment{a}\lstinline!.!\MyComment{b}\lstinline$f$
specifica che il valore di \lstinline$x$ va stampato in modo che occupi \emph{almeno} \MyComment{a} posizioni, di cui \MyComment{b} devono seguire il punto decimale. \\
		\bottomrule
	\end{tabular}
\end{table}
Per stampare a schermo (lo \emph{standard output} talvolta indicato con \lstinline!stdout!) si è utilizzata la funzione  \lstinline$printf()$,\index{printf()@\texttt{printf()}}\marginpar{\lstinline!printf()!} che ammette le sintassi riassunte nella tabella~\ref{tab:printf}.


Per leggere un valore inserito nello \emph{standard input}\index{standard input} (indicato con \lstinline!stdin!\index{stdin@\texttt{stdin}}), che nella maggior parte dei casi è la tastiera, si può usare la funzione \lstinline$scanf()$\index{scanf()@\texttt{scanf()}}\marginpar{\lstinline!scanf()!}.
\begin{table}
	\caption{Sintassi della funzione \lstinline[mathescape]?scanf($\text{\MyComment{entry}}$)?.}
	\label{tab:scanf}
	\centering
	\begin{tabular}{Fp{0.6\columnwidth}}
		\toprule
$\text{\MyComment{entry}}$ & Funzione \\
		\midrule
"\%d", \&n
& Legge un valore di tipo \lstinline!int! e lo assegna alla variabile \lstinline!n!, che dev'essere stata dichiarata di tipo intero. \\

"\%f", \&x
& Legge un valore decimale e lo assegna alla variabile \lstinline!x! di tipo \lstinline!float!. \\

"\%lf", \&y
& Legge un valore di tipo decimale e lo assegna alla variabile \lstinline!y! di tipo \lstinline!double!. \\
		\bottomrule
	\end{tabular}
\end{table}
Tale funzione ammette le sintassi elencate nella tabella~\ref{tab:scanf}.
In tutti i casi il carattere ``\lstinline!&!'' precede l'identificatore della variabile.


\begin{table}[p]
	\centering
	\caption[Opzioni di  \lstinline!printf()! e \lstinline!scanf()!]{Opzioni delle funzioni  \lstinline!printf()! e \lstinline!scanf()!}
	\label{tab:spop}
	\begin{tabular}{l p{0.75\columnwidth}}
		\toprule
Opzione	&Funzione	\\
		\midrule
\lstinline!%d!	& Stampa/legge variabili di tipo \lstinline!int!.		\\
\lstinline!%f!		& Stampa/legge variabili di tipo \lstinline!float!.		\\
\lstinline!%lf!	& Stampa/legge variabili di tipo \lstinline!double!.	\\
\lstinline!%c!		& Stampa/legge variabili di tipo \lstinline!char!.		\\
\lstinline!%s!		& (Con \lstinline!scanf()!)
			    Legge stringhe di caratteri e le salva in un array.			\\
		\bottomrule
	\end{tabular}
\end{table}

	\section{Strutture Dati}
Una \emph{struttura dati}\index{struttura dati} è un'entità usata per organizzare un insieme di dati all'interno della memoria del computer ed eventualmente per memorizzarli in un archivio di massa.\footnote{L'espressione <<struttura dati>>, ereditata dall'inglese \emph{data structures}, è accettata in italiano anche se grammaticalmente scorretta; la forma corretta sarebbe <<struttura dei dati>>.}
Detto in parole povere, è una rappresentazione schematica dei dati su cui s'intende lavorare.
La scelta delle strutture dati da utilizzare è strettamente legata a quella degli algoritmi.

		\subsection{Vettori}
		\label{subsec:array}
Il \emph{vettore}, o \emph{array}\index{array}, è un oggetto formato da una sequenza di valori omogenei (ossia, dello stesso tipo), una variabile in cui tutti i valori occupano posizioni adiacenti.
Se si deve risolvere un problema che viene naturalmente rappresentato sotto forma vettoriale (o matriciale) questa struttura dati risulta estremamente utile.


Per dichiarare un vettore di tipo \lstinline!double! e lunghezza tre, si usa la sintassi
\begin{lstlisting}
double p[3];
\end{lstlisting}
Di fatto questa dichiarazione corrisponde a quella di tre variabili, \lstinline!p[0]!, \lstinline!p[1]! e \lstinline!p[2]!, indicizzate da un intero non negativo che va da \numrange{0}{2} (cioè $3-1$).
Le variabili sono tutte di tipo \lstinline!double! ed occupano posizioni contigue in memoria condividendo una parte del nome, che è un \emph{puntatore}\index{puntatore} (paragrafo~\ref{sec:pointers} a pagina~\pageref{sec:pointers}) alla prima variabile dell'array---in altre parole \lstinline!p! e \lstinline!&p[0]! contengono lo stesso indirizzo.
Nulla vieta di utilizzare le tre \lstinline!p[!\MyComment{i}\lstinline!]! come normali variabili ma l'averle dichiarate sotto forma di array permette di manipolarle meglio per operazioni vettoriali.


Si supponga di aver bisogno di rappresentare la posizione di un punto rispetto all'origine di un sistema di assi cartesiani nello spazio tridimensionale.
La migliore approssimazione su un calcolatore di un numero reale è una variabile di tipo \lstinline!double! quindi la scelta naturale è usare un vettore composto da tre \lstinline!double! come quello di nome \lstinline!p! appena dichiarato.




Nel seguente codice esempio, dopo aver dichiarato un vettore di lunghezza \num{100}, s'assegna alla sua $e$-esima coordinata il valore $e^2-3e$.
\begin{lstlisting}
int v[100], e;
e = 0;
while ( e <= 99 ) {
	v[e] = e * e - 3 * e;
	e = e + 1;
}
\end{lstlisting}

	\section{Funzioni}
In matematica, una \emph{funzione} (o \emph{applicazione}) $f$ è una \emph{relazione} definita da:
\begin{itemize}
	\item
Un insieme $X$ detto \emph{Dominio} della funzione $f$;
	\item
Un insieme $Y$ detto \emph{Codominio} della funzione $f$;
	\item
Una legge tale per cui $\forall x \in X, \exists! \ y \in Y \mid y=f(x)$.
\end{itemize}

Si consideri la funzione \emph{valore assoluto} $\mathop{\mathrm{abs}}\colon\R\to\R^+\cup\Set{0}$ tale che $y = \mathop{\mathrm{abs}}(x)=\abs{x}$.
%\[
%y=
%\begin{sistema}
% \\
%\mathop{\mathrm{abs}}(x)=\abs{x}
%\end{sistema}
%\]
All'interno di un programma, si possono definire delle parti di codice che eseguono delle operazioni specifiche (\emph{funzioni}, appunto).
Per fare ciò, bisogna specificare un Dominio, un Codominio e la forma della funzione (la sequenza di istruzioni che la compongono).
Successivamente, la funzione sarà disponibile per essere richiamata in altre parti del programma.


\begin{lstlisting}[
	caption={Definizione e chiamata della funzine \lstinline!abs()!.},
	label={cod:absfunc},
	float
]
/* codominio nome_della_funzione ( dominio ) */
double abs ( double x ) {

	/* corpo di una funzione va sempre fra parentesi graffe */
	if ( x < 0 )
		return -x;
	else
		return x;
}

int main ( int argc, char *argv[] ) {
	double v, y;
	scanf( "%lf", &v );

	y = abs( v );£!\label{richiamo_abs}!£
	printf( "%lf", y );

	exit( EXIT_SUCCESS );
}
\end{lstlisting}
Il codice~\ref{cod:absfunc} mostra la precedente funzione stesa in linguaggio \lang{C}.
Si noti che, dopo averla dichiarata, la si può richiamare come a riga~\ref{richiamo_abs}.
Si tenga inoltre presente che ogni programma in \lang{C} \emph{deve} contenere una funzione \lstinline!main()!.


Si considerino delle funzioni con due (o più) parametri formali, il \emph{massimo} tra due numeri $\max\colon\R\x\R\to\R$ tale che 
\[
%\begin{sistema}
%\mathop{\mathrm{max}}\colon\mathbb{R}\times\mathbb{R}\mapsto\mathbb{R} \\
\mathop{\mathrm{max}}(x,y)\coloneqq
		\begin{cases}
x,	& \text{se } x>y \\
y,	& \text{se } x<y
		\end{cases}
%\end{sistema}
\]
%Essa trova il massimo tra due numeri.
\begin{lstlisting}[
	caption={Funzione \lstinline!max()! con due argomenti.},
	label={cod:MultiArgFunc},
	float
]
double max ( double x, double y ) {
	if ( x < y )
		return y;
	else
		return x;
}
\end{lstlisting}
Il codice~\ref{cod:MultiArgFunc}, ne mostra la stesura in \lang{C}.
