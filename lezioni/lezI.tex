\chapter[Lezione I]{Lezione I\newline\small{\emph{24/03/2011}}}
	\section{Introduzione informale all'informatica}
L'\emph{informatica} è una scienza che permette di risolvere per via algoritmica problemi come i seguenti.\footnote{Possiamo assumere, per il momento che l'algoritmo sia un processo meccanico che produca dei risultati a partire da dei dati seguendo dei passaggi prestabiliti.}


Si stabilisca quale percorso deve seguire un aereo per arrivare da un aeroporto ad un altro.
Qualora ci fossero più percorsi, si trovi quello più breve.
Tale problema si può risolvere schematizzando i percorsi e gli aeroporti.
Può tornare utile rappresentare gli aeroporti per mezzo di punti e le eventuali rotte che li collegano con dei segmenti.
Ora, non è più rilevante la struttura geografica dei luoghi o il percorso effettivamente seguito dal mezzo.
Basta conoscere gli aeroporti, le rotte di collegamento e il loro rispettivo ``peso'' (ossia la distanza).
Una rappresentazione come quella qui descritta si chiama \emph{grafo} (vedi il paragrafo~\vref{subsec:grafo});

Si è qui operato un processo di \emph{astrazione}. Esso consiste nell'estrapolare dal problema reale tutti e soli i dati che servano a risolverlo, trascurando dettagli inutili e semplificando così la questione.


Un ladro entra in un magazzino ed ha a disposizione solo uno zaino che riesce a sostenere un peso finito con cui trasportare la refurtiva.
Nel magazzino sono presenti oggetti di diverso valore e diverso peso.
Il suo scopo è di massimizzare il suo guadagno cercando di portare via oggetti del maggior valore possibile.
Tale questione si può esprimere in modo più formale come segue.
Sia dato uno zaino che possa sopportare un determinato peso $W$.
Siano dati inoltre $N$ oggetti, ognuno dei quali caratterizzato da un peso $w_i$ e un valore $c_i$ con $i\in\{1,\dots,N\}$.
Si scelga quali di questi oggetti mettere nello zaino per ottenere il maggiore valore senza eccedere nel peso sostenibile dallo zaino stesso \cite{wiki:zaino}.

Una soluzione possibile consiste nello scegliere gli oggetti da portare via in base al loro valore unitario, cioè al rapporto:
\[
VU_i=\frac{c_i}{w_i}.
\]
Ovviamente, un oggetto avrà priorità tanto maggiore quanto più grande sarà il suo valore unitario.


Si hanno \num{50} schedine numerate, indicizzate da numeri \emph{non} progressivi compresi in un range che va da \numrange{1}{e6}. Bisogna ridisporre tale schedine in ordine crescente. In particolare, ci sono dei vincoli. Supponiamo di avere a disposizione uno spazio che non permetta di vedere i numeri di tutte le schedine contemporaneamente. C'è, però, a abbastanza spazio per appoggiare contemporaneamente due mazzi.

Un procedimento risolutivo potrebbe essere il seguente. Si estrae una schedina $s_1$ dal mazzo non ordinato e la si appoggia accanto al mazzo stesso, in modo che il suo numero sia visibile. Se n'estrae una seconda $s_2$ e la si confronta con la prima. Se $s_1>s_2$, si pone $s_2$ sopra $s_1$. Altrimenti si ripone $s_2$ in fondo al mazzo da cui è stata estratta. Si ripete tale procedimento fino a quando il primo mazzo non finisce. In informatica, tale metodo risolutivo prende talvolta il nome di \emph{bubble sort}.



		\subsection{Efficienza di un algoritmo}
		\label{subsec:eff}
Quando si stende un algoritmo, bisogna cercare di renderlo il più efficiente e performante possibile.
L'efficienza di un algoritmo, come ci si potrebbe aspettare, è inversamente proporzionale al numero di ``calcoli'' che l'esecutore deve compiere per giungere al risultato. 

La \marginpar{bubble sort} tabella~\vref{tab:bubble} mostra il numero massimo di confronti da compiere (in riferimento alla terza tipologia di problema) tramite un algoritmo di bubble sort in relazione al numero di carte che rimangono nel primo mazzo. In generale, supponendo di avere un mazzo non ordinato di $n$ carte, con $n\in\mathbb{N}$, si hanno $(n-1)+(n-2)+\dots+2+1$ confronti $c$, ossia:
\[
c_\textup{bs}(n)=\sum_{i=1}^{n-1}i=\frac{(n-2)(n-1)}{2}
\]
dove, l'ultima uguaglianza è giustificata dall'algoritmo di Gauss per la somma dei primi $n$ numeri naturali. Pertanto, per $n\to+\infty$, si ha che:
\[
c_\textup{bs}(n)
= \sum_{i=1}^{n-1}i=\frac{(n-2)(n-1)}{2}
\overset{n\to\infty}{\thicksim}
\frac{\ n^2}{2}\asymp n^{2}.
\]
Raddoppiando il numero di carte, dunque, il numero di confronti quadruplica.

Esistono \marginpar{merge sort} molti altri metodi d'ordinamento, tra cui l'algoritmo \emph{merge sort}, inventato da John von Neumann nel 1945, che si basa un po' sull'antico detto romano ``divide et impera'' \parencite{wiki:it}.
Come mostrato in figura~\ref{fig:merge}, si dividono progressivamente i dati di partenza (nel nostro caso, le \num{50} schedine) in due gruppi fino ad ottenerne mazzi da due o un dato.
Se il numero di dati non è divisibile per due, non importa: si avrà almeno un mazzo da una schedina.
Dopo aver suddiviso le schedine, si ordinano progressivamente partendo ``dal basso verso l'alto''.

Per quanto riguarda il costo di tale algoritmo, si devono effettuare all'incirca $\log_2(n)$ suddivisioni.
Ad ogni passaggio a ritroso, inoltre, bisogna compiere $n$ confronti.
Il costo dell'algoritmo sarà, quindi, $c_\textup{ms}(n)\thicksim n\log_{2}n$. Per $n\to+\infty$, si ha che:
\[
c_\textup{bs}(n)\overset{n\to+\infty}{\thicksim} n^{2} \gg n\log_{2}(n) \overset{n\to+\infty}{\thicksim} c_\textup{ms}(n).
\]
Si riscontra, quindi, che l'algoritmo merge sort è più efficiente dell'algoritmo bubble sort.
\begin{table}
	\centering
	\caption[Costo dell'algoritmo bubble sort]{Costo computazionale di un algoritmo bubble sort ($n\in\mathbb{N}\sm\Set{0}$).}
	\label{tab:bubble}
	\begin{tabular}{>$c<$ >$c<$}
		\toprule
\emph{n. di carte}	&\emph{n. di confronti}	\\
		\midrule
50			&49				\\
49			&48				\\
48			&47				\\
\vdots 		&\vdots 			\\
n			&n-1				\\
		\bottomrule
	\end{tabular}
\end{table}

\begin{figure}
	\centering
	\includegraphics[width=\columnwidth]{immagini/merge_sort}
	\caption[Merge sort]{Rappresentazione dell'algoritmo merge sort.}
	\label{fig:merge}
\end{figure}

	\section{Programma}
Il \emph{programma} è un oggetto che si può presentare sotto varie forme. Le più importanti sono tre:
\begin{description}
	\item[In corso d'esecuzione.] In questo caso, prende il nome di \emph{processo};
	\item[File eseguibile.] Esso è essenzialmente un file (con estensione \ext{.exe} in \os{Windows}) contenente un insieme d'istruzioni comprensibili dall'unità di calcolo. Una volta eseguito, si traduce in un processo. Il linguaggio in cui è scritto tale file prende il nome di \emph{assembly} o \emph{linguaggio macchina}. Quest'ultimo nome lascia intuire che tale file non è fatto per essere letto ed interpretato dall'uomo;
	\item[Codice sorgente.] Questo è un file di testo contenente istruzioni comprensibili anche all'uomo. \`E scritto in un \emph{linguaggio di programmazione} (vedi il paragrafo~\ref{subsec:programmazione} a pagina~\pageref{subsec:programmazione}). Per ottenere un file eseguibile dal codice sorgente, c'è bisogno di ``tradurre'' quest'ultimo tramite un processo che, per ragioni storiche, prende il nome di \emph{compilazione}.\footnote{Per maggiori dettagli, si rimanda a degli appunti sul sito del prof.~Bernardinello all'indirizzo \url{http://www.mc3.disco.unimib.it/lif/Dep/vigcc.pdf}}.
\end{description}